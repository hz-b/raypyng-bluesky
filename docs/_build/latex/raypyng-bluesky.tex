%% Generated by Sphinx.
\def\sphinxdocclass{report}
\documentclass[letterpaper,10pt,english]{sphinxmanual}
\ifdefined\pdfpxdimen
   \let\sphinxpxdimen\pdfpxdimen\else\newdimen\sphinxpxdimen
\fi \sphinxpxdimen=.75bp\relax
\ifdefined\pdfimageresolution
    \pdfimageresolution= \numexpr \dimexpr1in\relax/\sphinxpxdimen\relax
\fi
%% let collapsible pdf bookmarks panel have high depth per default
\PassOptionsToPackage{bookmarksdepth=5}{hyperref}

\PassOptionsToPackage{warn}{textcomp}
\usepackage[utf8]{inputenc}
\ifdefined\DeclareUnicodeCharacter
% support both utf8 and utf8x syntaxes
  \ifdefined\DeclareUnicodeCharacterAsOptional
    \def\sphinxDUC#1{\DeclareUnicodeCharacter{"#1}}
  \else
    \let\sphinxDUC\DeclareUnicodeCharacter
  \fi
  \sphinxDUC{00A0}{\nobreakspace}
  \sphinxDUC{2500}{\sphinxunichar{2500}}
  \sphinxDUC{2502}{\sphinxunichar{2502}}
  \sphinxDUC{2514}{\sphinxunichar{2514}}
  \sphinxDUC{251C}{\sphinxunichar{251C}}
  \sphinxDUC{2572}{\textbackslash}
\fi
\usepackage{cmap}
\usepackage[T1]{fontenc}
\usepackage{amsmath,amssymb,amstext}
\usepackage{babel}



\usepackage{tgtermes}
\usepackage{tgheros}
\renewcommand{\ttdefault}{txtt}



\usepackage[Bjarne]{fncychap}
\usepackage{sphinx}

\fvset{fontsize=auto}
\usepackage{geometry}


% Include hyperref last.
\usepackage{hyperref}
% Fix anchor placement for figures with captions.
\usepackage{hypcap}% it must be loaded after hyperref.
% Set up styles of URL: it should be placed after hyperref.
\urlstyle{same}

\addto\captionsenglish{\renewcommand{\contentsname}{Contents:}}

\usepackage{sphinxmessages}
\setcounter{tocdepth}{2}



\title{raypyng\sphinxhyphen{}bluesky}
\date{Dec 02, 2022}
\release{}
\author{Simone Vadilonga, Ruslan Ovsyannikov}
\newcommand{\sphinxlogo}{\vbox{}}
\renewcommand{\releasename}{}
\makeindex
\begin{document}

\ifdefined\shorthandoff
  \ifnum\catcode`\=\string=\active\shorthandoff{=}\fi
  \ifnum\catcode`\"=\active\shorthandoff{"}\fi
\fi

\pagestyle{empty}
\sphinxmaketitle
\pagestyle{plain}
\sphinxtableofcontents
\pagestyle{normal}
\phantomsection\label{\detokenize{index::doc}}


\sphinxAtStartPar
This is a raypyng\sphinxhyphen{}Bluesky interface library that allows using the Bluesky data acquisition framework to run RAY\sphinxhyphen{}UI simulations.
It can be used to implement a digital twin at the beamline.
This package is based on Bluesky, raypyng and RAY\sphinxhyphen{}UI.

\sphinxAtStartPar
\sphinxhref{https://blueskyproject.io/}{Bluesky}
\sphinxhref{https://raypyng-bluesky.readthedocs.io/en/latest/index.html}{Raypyng}
\sphinxhref{https://www.helmholtz-berlin.de/forschung/oe/wi/optik-strahlrohre/arbeitsgebiete/ray\_en.html}{Ray\sphinxhyphen{}UI}

\sphinxstepscope


\chapter{Installation}
\label{\detokenize{installation:installation}}\label{\detokenize{installation::doc}}
\sphinxAtStartPar
raypyng\sphinxhyphen{}bluesky will work only if using a Linux or a macOS distribution.


\section{Install RAY\sphinxhyphen{}UI}
\label{\detokenize{installation:install-ray-ui}}
\sphinxAtStartPar
Download the RAY\sphinxhyphen{}UI installer from  \sphinxhref{https://www.helmholtz-berlin.de/forschung/oe/wi/optik-strahlrohre/arbeitsgebiete/ray\_en.html}{this link},
and run the installer.


\section{Install xvfb}
\label{\detokenize{installation:install-xvfb}}
\sphinxAtStartPar
xvfb is a virtual X11 framebuffer server that let you run RAY\sphinxhyphen{}UI headless

\sphinxAtStartPar
Install xvfb:

\begin{sphinxVerbatim}[commandchars=\\\{\}]
sudo apt install xvfb
\end{sphinxVerbatim}

\begin{sphinxadmonition}{note}{Note:}
\sphinxAtStartPar
xvfb\sphinxhyphen{}run script is a part of the xvfb distribution and
runs an app on a new virtual X11 server.
\end{sphinxadmonition}


\section{Install raypyng\sphinxhyphen{}bluesky}
\label{\detokenize{installation:install-raypyng-bluesky}}\begin{itemize}
\item {} 
\sphinxAtStartPar
You will need Python 3.8 or newer. From a shell (“Terminal” on OSX),
check your current Python version.

\begin{sphinxVerbatim}[commandchars=\\\{\}]
python3 \PYGZhy{}\PYGZhy{}version
\end{sphinxVerbatim}

\sphinxAtStartPar
If that version is less than 3.8, you must update it.

\sphinxAtStartPar
We recommend installing raypyng into a “virtual environment” so that this
installation will not interfere with any existing Python software:

\begin{sphinxVerbatim}[commandchars=\\\{\}]
python3 \PYGZhy{}m venv \PYGZti{}/raypyng\PYGZhy{}bluesky\PYGZhy{}tutorial
\PYG{n+nb}{source} \PYGZti{}/raypyng\PYGZhy{}bluesky\PYGZhy{}tutorial/bin/activate
\end{sphinxVerbatim}

\sphinxAtStartPar
Alternatively, if you are a
\sphinxhref{https://conda.io/docs/user-guide/install/download.html}{conda} user,
you can create a conda environment:

\begin{sphinxVerbatim}[commandchars=\\\{\}]
conda create \PYGZhy{}n raypyng\PYGZhy{}bluesky\PYGZhy{}tutorial \PYG{l+s+s2}{\PYGZdq{}python\PYGZgt{}=3.8\PYGZdq{}}
conda activate raypyng\PYGZhy{}bluesky\PYGZhy{}tutorial
\end{sphinxVerbatim}

\item {} 
\sphinxAtStartPar
Install the latest versions of raypyng and ophyd. Also, install IPython
(a Python interpreter designed by scientists for scientists).

\begin{sphinxVerbatim}[commandchars=\\\{\}]
python3 \PYGZhy{}m pip install \PYGZhy{}\PYGZhy{}upgrade raypyng\PYGZhy{}bluesky ipython
\end{sphinxVerbatim}

\item {} 
\sphinxAtStartPar
Start IPython:

\begin{sphinxVerbatim}[commandchars=\\\{\}]
\PYG{n}{ipython} \PYG{o}{\PYGZhy{}}\PYG{o}{\PYGZhy{}}\PYG{n}{matplotlib}\PYG{o}{=}\PYG{n}{qt5}
\end{sphinxVerbatim}

\sphinxAtStartPar
The flag \sphinxcode{\sphinxupquote{\sphinxhyphen{}\sphinxhyphen{}matplotlib=qt5}} is necessary for live\sphinxhyphen{}updating plots to work.

\sphinxAtStartPar
Or, if you wish you use raypyng from a Jupyter notebook, install a kernel like
so:

\begin{sphinxVerbatim}[commandchars=\\\{\}]
\PYG{n}{ipython} \PYG{n}{kernel} \PYG{n}{install} \PYG{o}{\PYGZhy{}}\PYG{o}{\PYGZhy{}}\PYG{n}{user} \PYG{o}{\PYGZhy{}}\PYG{o}{\PYGZhy{}}\PYG{n}{name}\PYG{o}{=}\PYG{n}{raypyng}\PYG{o}{\PYGZhy{}}\PYG{n}{bluesky}\PYG{o}{\PYGZhy{}}\PYG{n}{tutorial} \PYG{o}{\PYGZhy{}}\PYG{o}{\PYGZhy{}}\PYG{n}{display}\PYG{o}{\PYGZhy{}}\PYG{n}{name} \PYG{l+s+s2}{\PYGZdq{}}\PYG{l+s+s2}{Python (raypyng\PYGZhy{}bluesky)}\PYG{l+s+s2}{\PYGZdq{}}
\end{sphinxVerbatim}

\sphinxAtStartPar
You may start Jupyter from any environment where it is already installed, or
install it in this environment alongside raypyng and run it from there:

\begin{sphinxVerbatim}[commandchars=\\\{\}]
\PYG{n}{pip} \PYG{n}{install} \PYG{n}{notebook}
\PYG{n}{jupyter} \PYG{n}{notebook}
\end{sphinxVerbatim}

\end{itemize}

\sphinxstepscope


\chapter{Tutorial}
\label{\detokenize{tutorial:tutorial}}\label{\detokenize{tutorial::doc}}

\section{Setup an Ipython profile}
\label{\detokenize{tutorial:setup-an-ipython-profile}}
\sphinxAtStartPar
The code is thought to be used in an environment where bluesky is setup. For doing this it is convenient to create an ipython profile
and modify the startup files. An rml file created with RAY\sphinxhyphen{}UI containg a beamline is also needed.
The code in the following example and an rml file ready to use is available at this \sphinxhref{https://github.com/hz-b/raypyng-bluesky/tree/main/examples/profile\_raypyng-bluesky-tutorial}{link}
In the startup folder of the ipython profile create a file called \sphinxcode{\sphinxupquote{0\sphinxhyphen{}bluesky.py}} that contains a minimal setup of bluesky and raypyng\sphinxhyphen{}bluesky.

\sphinxAtStartPar
The first part of the file contains a minimal installation of bluesky

\begin{sphinxVerbatim}[commandchars=\\\{\}]
\PYG{k+kn}{import} \PYG{n+nn}{os}
\PYG{k+kn}{from} \PYG{n+nn}{bluesky} \PYG{k+kn}{import} \PYG{n}{RunEngine}
\PYG{k+kn}{from} \PYG{n+nn}{raypyng\PYGZus{}bluesky}\PYG{n+nn}{.}\PYG{n+nn}{RaypyngOphydDevices} \PYG{k+kn}{import} \PYG{n}{RaypyngOphydDevices}

\PYG{n}{RE} \PYG{o}{=} \PYG{n}{RunEngine}\PYG{p}{(}\PYG{p}{\PYGZob{}}\PYG{p}{\PYGZcb{}}\PYG{p}{)}

\PYG{c+c1}{\PYGZsh{} Send all metadata/data captured to the BestEffortCallback.}
\PYG{k+kn}{from} \PYG{n+nn}{bluesky}\PYG{n+nn}{.}\PYG{n+nn}{callbacks}\PYG{n+nn}{.}\PYG{n+nn}{best\PYGZus{}effort} \PYG{k+kn}{import} \PYG{n}{BestEffortCallback}
\PYG{n}{bec} \PYG{o}{=} \PYG{n}{BestEffortCallback}\PYG{p}{(}\PYG{p}{)}


\PYG{c+c1}{\PYGZsh{} Make plots update live while scans run.}
\PYG{k+kn}{from} \PYG{n+nn}{bluesky}\PYG{n+nn}{.}\PYG{n+nn}{utils} \PYG{k+kn}{import} \PYG{n}{install\PYGZus{}kicker}
\PYG{n}{install\PYGZus{}kicker}\PYG{p}{(}\PYG{p}{)}

\PYG{c+c1}{\PYGZsh{} create a temporary database}
\PYG{k+kn}{from} \PYG{n+nn}{databroker} \PYG{k+kn}{import} \PYG{n}{Broker}
\PYG{n}{db} \PYG{o}{=} \PYG{n}{Broker}\PYG{o}{.}\PYG{n}{named}\PYG{p}{(}\PYG{l+s+s1}{\PYGZsq{}}\PYG{l+s+s1}{temp}\PYG{l+s+s1}{\PYGZsq{}}\PYG{p}{)}
\PYG{n}{RE}\PYG{o}{.}\PYG{n}{subscribe}\PYG{p}{(}\PYG{n}{db}\PYG{o}{.}\PYG{n}{insert}\PYG{p}{)}
\PYG{n}{RE}\PYG{o}{.}\PYG{n}{subscribe}\PYG{p}{(}\PYG{n}{bec}\PYG{p}{)}

\PYG{c+c1}{\PYGZsh{} import the magics}
\PYG{k+kn}{from} \PYG{n+nn}{bluesky}\PYG{n+nn}{.}\PYG{n+nn}{magics} \PYG{k+kn}{import} \PYG{n}{BlueskyMagics}
\PYG{n}{get\PYGZus{}ipython}\PYG{p}{(}\PYG{p}{)}\PYG{o}{.}\PYG{n}{register\PYGZus{}magics}\PYG{p}{(}\PYG{n}{BlueskyMagics}\PYG{p}{)}

\PYG{c+c1}{\PYGZsh{} import plans}
\PYG{k+kn}{from} \PYG{n+nn}{bluesky}\PYG{n+nn}{.}\PYG{n+nn}{plans} \PYG{k+kn}{import} \PYG{p}{(}
    \PYG{n}{relative\PYGZus{}scan} \PYG{k}{as} \PYG{n}{dscan}\PYG{p}{,}
    \PYG{n}{scan}\PYG{p}{,} \PYG{n}{scan} \PYG{k}{as} \PYG{n}{ascan}\PYG{p}{,}
    \PYG{n}{list\PYGZus{}scan}\PYG{p}{,}
    \PYG{n}{rel\PYGZus{}list\PYGZus{}scan}\PYG{p}{,}
    \PYG{n}{rel\PYGZus{}grid\PYGZus{}scan}\PYG{p}{,}  \PYG{n}{rel\PYGZus{}grid\PYGZus{}scan} \PYG{k}{as} \PYG{n}{dmesh}\PYG{p}{,}
    \PYG{n}{list\PYGZus{}grid\PYGZus{}scan}\PYG{p}{,}
    \PYG{n}{adaptive\PYGZus{}scan}\PYG{p}{,}
    \PYG{n}{rel\PYGZus{}adaptive\PYGZus{}scan}\PYG{p}{,}
    \PYG{n}{inner\PYGZus{}product\PYGZus{}scan}            \PYG{k}{as} \PYG{n}{a2scan}\PYG{p}{,}
    \PYG{n}{relative\PYGZus{}inner\PYGZus{}product\PYGZus{}scan}   \PYG{k}{as} \PYG{n}{d2scan}\PYG{p}{,}
    \PYG{n}{tweak}\PYG{p}{)}

\PYG{c+c1}{\PYGZsh{} import stubs}
\PYG{k+kn}{from} \PYG{n+nn}{bluesky}\PYG{n+nn}{.}\PYG{n+nn}{plan\PYGZus{}stubs} \PYG{k+kn}{import} \PYG{p}{(}
    \PYG{n}{abs\PYGZus{}set}\PYG{p}{,}\PYG{n}{rel\PYGZus{}set}\PYG{p}{,}
    \PYG{n}{mv}\PYG{p}{,} \PYG{n}{mvr}\PYG{p}{,}
    \PYG{n}{trigger}\PYG{p}{,}
    \PYG{n}{read}\PYG{p}{,} \PYG{n}{rd}\PYG{p}{,}
    \PYG{n}{stage}\PYG{p}{,} \PYG{n}{unstage}\PYG{p}{,}
    \PYG{n}{configure}\PYG{p}{,}
    \PYG{n}{stop}\PYG{p}{)}
\end{sphinxVerbatim}

\sphinxAtStartPar
The last part contains the the two lines of code used to create RaypyngOphyd devices. See the API documentation for
more details about \sphinxcode{\sphinxupquote{RaypyngOphydDevices}}. If you already have an ipython profile with Bluesky you can just add these lines.

\begin{sphinxVerbatim}[commandchars=\\\{\}]
\PYG{c+c1}{\PYGZsh{} insert here the path to the rml file that you want to use}
\PYG{n}{rml\PYGZus{}path} \PYG{o}{=} \PYG{l+s+s1}{\PYGZsq{}}\PYG{l+s+s1}{...rml/elisa.rml}\PYG{l+s+s1}{\PYGZsq{}}

\PYG{n}{RaypyngOphydDevices}\PYG{p}{(}\PYG{n}{RE}\PYG{o}{=}\PYG{n}{RE}\PYG{p}{,} \PYG{n}{rml\PYGZus{}path}\PYG{o}{=}\PYG{n}{rml\PYGZus{}path}\PYG{p}{,} \PYG{n}{temporary\PYGZus{}folder}\PYG{o}{=}\PYG{k+kc}{None}\PYG{p}{,} \PYG{n}{name\PYGZus{}space}\PYG{o}{=}\PYG{k+kc}{None}\PYG{p}{,} \PYG{n}{prefix}\PYG{o}{=}\PYG{k+kc}{None}\PYG{p}{,} \PYG{n}{ray\PYGZus{}ui\PYGZus{}location}\PYG{o}{=}\PYG{k+kc}{None}\PYG{p}{)}
\end{sphinxVerbatim}

\sphinxAtStartPar
The ipython profile can be started using:

\begin{sphinxVerbatim}[commandchars=\\\{\}]
\PYG{n}{ipython} \PYG{o}{\PYGZhy{}}\PYG{o}{\PYGZhy{}}\PYG{n}{profile}\PYG{o}{=}\PYG{n}{raypyng}\PYG{o}{\PYGZhy{}}\PYG{n}{bluesky}\PYG{o}{\PYGZhy{}}\PYG{n}{tutorial} \PYG{o}{\PYGZhy{}}\PYG{o}{\PYGZhy{}}\PYG{n}{matplotlib}\PYG{o}{=}\PYG{n}{qt5}
\end{sphinxVerbatim}

\sphinxAtStartPar
All the elements present in the rml file as ophyd devices. If you set \sphinxcode{\sphinxupquote{prefix=None}}, the prefix \sphinxcode{\sphinxupquote{rp\_}} is automatically
prepended to the name of the optical elements found in the rml file to create the dame of the object in python. If you have a Dipole called
\sphinxcode{\sphinxupquote{Dipole}}, then the name would be: \sphinxcode{\sphinxupquote{rp\_Dipole}}. You can now use the simulated motors as you would normally do in bluesky.

\sphinxAtStartPar
To see a list of the implemented motors and detectors use the ipython autocompletion by typing in the ipython shell

\begin{sphinxVerbatim}[commandchars=\\\{\}]
\PYG{n}{rp\PYGZus{}}
\end{sphinxVerbatim}

\sphinxAtStartPar
and pressing \sphinxcode{\sphinxupquote{tab}}.


\section{What can go wrong, and how to correct it}
\label{\detokenize{tutorial:what-can-go-wrong-and-how-to-correct-it}}
\sphinxAtStartPar
If once you setup the ipython profile as explained in the section above no elements are found, might be that the \sphinxcode{\sphinxupquote{RaypyngOphydDevices}}
class fails to insert the ophyd devices in the correct namespace. In this case try to call the classes passing explicitly the correct namespace
like this:

\begin{sphinxVerbatim}[commandchars=\\\{\}]
\PYG{k+kn}{import} \PYG{n+nn}{sys}
\PYG{n}{RaypyngOphydDevices}\PYG{p}{(}\PYG{n}{RE}\PYG{o}{=}\PYG{n}{RE}\PYG{p}{,} \PYG{n}{rml\PYGZus{}path}\PYG{o}{=}\PYG{n}{rml\PYGZus{}path}\PYG{p}{,} \PYG{n}{temporary\PYGZus{}folder}\PYG{o}{=}\PYG{k+kc}{None}\PYG{p}{,} \PYG{n}{name\PYGZus{}space}\PYG{o}{=}\PYG{n}{sys}\PYG{o}{.}\PYG{n}{\PYGZus{}getframe}\PYG{p}{(}\PYG{l+m+mi}{0}\PYG{p}{)}\PYG{p}{,} \PYG{n}{prefix}\PYG{o}{=}\PYG{k+kc}{None}\PYG{p}{,} \PYG{n}{ray\PYGZus{}ui\PYGZus{}location}\PYG{o}{=}\PYG{k+kc}{None}\PYG{p}{)}
\end{sphinxVerbatim}

\sphinxAtStartPar
If when you start a scan (see section below in this tutorial), RAY\sphinxhyphen{}UI is not found, it is because you installed it in a non\sphinxhyphen{}standard location.
In this case simply pass the absolute path of the folder where you installed RAY\sphinxhyphen{}UI to the class:

\begin{sphinxVerbatim}[commandchars=\\\{\}]
\PYG{n}{ray\PYGZus{}path} \PYG{o}{=} \PYG{o}{.}\PYG{o}{.}\PYG{o}{.} \PYG{c+c1}{\PYGZsh{} here the path to RAY\PYGZhy{}UI folder}
\PYG{n}{RaypyngOphydDevices}\PYG{p}{(}\PYG{n}{RE}\PYG{o}{=}\PYG{n}{RE}\PYG{p}{,} \PYG{n}{rml\PYGZus{}path}\PYG{o}{=}\PYG{n}{rml\PYGZus{}path}\PYG{p}{,} \PYG{n}{temporary\PYGZus{}folder}\PYG{o}{=}\PYG{k+kc}{None}\PYG{p}{,} \PYG{n}{name\PYGZus{}space}\PYG{o}{=}\PYG{k+kc}{None}\PYG{p}{,} \PYG{n}{prefix}\PYG{o}{=}\PYG{k+kc}{None}\PYG{p}{,} \PYG{n}{ray\PYGZus{}ui\PYGZus{}location}\PYG{o}{=}\PYG{n}{ray\PYGZus{}path}\PYG{p}{)}
\end{sphinxVerbatim}


\section{RaypyngOphyd \sphinxhyphen{} Motors}
\label{\detokenize{tutorial:raypyngophyd-motors}}
\sphinxAtStartPar
Presently only a subset of the parameters available in rml file in RAY\sphinxhyphen{}UI are implemented as motor axes. To see which ones are available,
use the tab\sphinxhyphen{}autocompletion. For instance, to see what axes are available for the motor \sphinxcode{\sphinxupquote{rp\_Dipole}} write in the ipython shell:

\begin{sphinxVerbatim}[commandchars=\\\{\}]
\PYG{n}{rp\PYGZus{}Dipole}\PYG{o}{.}
\end{sphinxVerbatim}

\sphinxAtStartPar
and press tab: among the other things you will see that are implemented \sphinxcode{\sphinxupquote{rp\_Dipole.nrays}}, the number of rays to use in the simulation,
and \sphinxcode{\sphinxupquote{p\_Dipole.en}}, the photon energy in eV. You can of course also use the \sphinxcode{\sphinxupquote{.get()}} and \sphinxcode{\sphinxupquote{.set()}} methods:

\begin{sphinxVerbatim}[commandchars=\\\{\}]
\PYG{n}{In} \PYG{p}{[}\PYG{l+m+mi}{1}\PYG{p}{]}\PYG{p}{:} \PYG{n}{rp\PYGZus{}Dipole}\PYG{o}{.}\PYG{n}{en}\PYG{o}{.}\PYG{n}{get}\PYG{p}{(}\PYG{p}{)}
\PYG{n}{Out}\PYG{p}{[}\PYG{l+m+mi}{1}\PYG{p}{]}\PYG{p}{:} \PYG{l+m+mf}{1000.0}

\PYG{n}{In} \PYG{p}{[}\PYG{l+m+mi}{2}\PYG{p}{]}\PYG{p}{:} \PYG{n}{rp\PYGZus{}Dipole}\PYG{o}{.}\PYG{n}{en}\PYG{o}{.}\PYG{n}{set}\PYG{p}{(}\PYG{l+m+mi}{1500}\PYG{p}{)}
\PYG{n}{Out}\PYG{p}{[}\PYG{l+m+mi}{2}\PYG{p}{]}\PYG{p}{:} \PYG{o}{\PYGZlt{}}\PYG{n}{ophyd}\PYG{o}{.}\PYG{n}{sim}\PYG{o}{.}\PYG{n}{NullStatus} \PYG{n}{at} \PYG{l+m+mh}{0x7fbf4c25adc0}\PYG{o}{\PYGZgt{}}

\PYG{n}{In} \PYG{p}{[}\PYG{l+m+mi}{3}\PYG{p}{]}\PYG{p}{:} \PYG{n}{rp\PYGZus{}Dipole}\PYG{o}{.}\PYG{n}{en}\PYG{o}{.}\PYG{n}{get}\PYG{p}{(}\PYG{p}{)}
\PYG{n}{Out}\PYG{p}{[}\PYG{l+m+mi}{3}\PYG{p}{]}\PYG{p}{:} \PYG{l+m+mf}{1500.0}
\end{sphinxVerbatim}

\sphinxAtStartPar
For a complete description of the axis available for each optical element see the \sphinxhref{https://raypyng-bluesky.readthedocs.io/en/latest/API.html\#id1}{API documentation}


\section{RaypyngOphyd \sphinxhyphen{} Detectors}
\label{\detokenize{tutorial:raypyngophyd-detectors}}
\sphinxAtStartPar
When an \sphinxcode{\sphinxupquote{ImagePlane}}, or an \sphinxcode{\sphinxupquote{ImagePlaneBundle}} is found in the rml file, a detector is created. Each detector
can return four properties of the x\sphinxhyphen{}ray beam. For instance, for the \sphinxcode{\sphinxupquote{DetectorAtFocus}}:
\sphinxhyphen{} \sphinxcode{\sphinxupquote{rp\_DetectorAtFocus.intensity}}: the intensity {[}Ph/s/A/BW{]}
\sphinxhyphen{} \sphinxcode{\sphinxupquote{rp\_DetectorAtFocus.bw}}: the bandwidth  {[}eV{]}
\sphinxhyphen{} \sphinxcode{\sphinxupquote{rp\_DetectorAtFocus.hor\_foc}}: the horizontal focus {[}um{]}
\sphinxhyphen{} \sphinxcode{\sphinxupquote{rp\_DetectorAtFocus.ver\_foc}}: the vertical focus {[}um{]}


\section{A scan in Bluesky}
\label{\detokenize{tutorial:a-scan-in-bluesky}}
\sphinxAtStartPar
It is possible to do scan using the simulation engine RAY\sphinxhyphen{}UI as it is normally done in bluesky.
For instance you can scan the photon energy and see the intensity at the source and and the sample position.
While at the beamline to change the energy we would simply ask the monochromator to do it, for the simulations
one needs to change the energy of the source

\begin{sphinxVerbatim}[commandchars=\\\{\}]
\PYG{n}{RE}\PYG{p}{(}\PYG{n}{scan}\PYG{p}{(}\PYG{p}{[}\PYG{n}{rp\PYGZus{}DetectorAtSource}\PYG{o}{.}\PYG{n}{intensity}\PYG{p}{,}\PYG{n}{rp\PYGZus{}DetectorAtFocus}\PYG{o}{.}\PYG{n}{intensity}\PYG{p}{]}\PYG{p}{,} \PYG{n}{rp\PYGZus{}Dipole}\PYG{o}{.}\PYG{n}{en}\PYG{p}{,} \PYG{l+m+mi}{200}\PYG{p}{,} \PYG{l+m+mi}{2200}\PYG{p}{,} \PYG{l+m+mi}{11}\PYG{p}{)}\PYG{p}{)}
\end{sphinxVerbatim}

\sphinxstepscope


\chapter{How To Guides}
\label{\detokenize{how_to:how-to-guides}}\label{\detokenize{how_to::doc}}

\section{Change grating}
\label{\detokenize{how_to:change-grating}}
\sphinxAtStartPar
This feature is stil experimental, and the implementation is somehow poor. However, the method can still be used to implement a grating change.

\sphinxAtStartPar
When the \sphinxcode{\sphinxupquote{RaypyngOphydDevices}} class is called, Ophyd devices are automatically created
.. code:: python
\begin{quote}

\sphinxAtStartPar
from raypyng\_bluesky.RaypyngOphydDevices import RaypyngOphydDevices

\sphinxAtStartPar
\# define here the path to the rml file
rml\_path = (’…rml/elisa.rml’)

\sphinxAtStartPar
RaypyngOphydDevices(RE=RE, rml\_path=rml\_path, temporary\_folder=None, name\_space=None, ray\_ui\_location=’/home/simone/RAY\sphinxhyphen{}UI\sphinxhyphen{}development’)\#sys.\_getframe(0))
\end{quote}

\sphinxAtStartPar
In this case we know that inside the \sphinxcode{\sphinxupquote{elisa.rml}} file we have a plane grating monochromator, and an elment that is the a plane grating called \sphinxcode{\sphinxupquote{PG}},
and that an Ophyd device called \sphinxcode{\sphinxupquote{rp\_PG}} is therefore created. The gratings can hold any number of different configurations, and the configuration found in the
rml file is saved with the name \sphinxcode{\sphinxupquote{\textquotesingle{}default\textquotesingle{}}}
We can rename the default grating to a more meaningful name, in this case since it is a 1200 lines/mm blazed grating we will call it simply ‘1200’
.. code:: python
\begin{quote}

\sphinxAtStartPar
rp\_PG.rename\_default\_grating(‘1200’)
\end{quote}

\sphinxAtStartPar
the second grating is a laminar grating with a pitch of 400 lines/mm.

\begin{sphinxVerbatim}[commandchars=\\\{\}]
\PYG{n}{rrp\PYGZus{}PG}\PYG{o}{.}\PYG{n}{gratings}\PYG{o}{=}\PYG{p}{(}\PYG{l+s+s1}{\PYGZsq{}}\PYG{l+s+s1}{400}\PYG{l+s+s1}{\PYGZsq{}}\PYG{p}{,} \PYG{p}{\PYGZob{}}\PYG{l+s+s1}{\PYGZsq{}}\PYG{l+s+s1}{lineDensity}\PYG{l+s+s1}{\PYGZsq{}}\PYG{p}{:}\PYG{l+m+mi}{400}\PYG{p}{,}
                    \PYG{l+s+s1}{\PYGZsq{}}\PYG{l+s+s1}{orderDiffraction}\PYG{l+s+s1}{\PYGZsq{}}\PYG{p}{:}\PYG{l+m+mi}{1}\PYG{p}{,}
                    \PYG{l+s+s1}{\PYGZsq{}}\PYG{l+s+s1}{lineProfile}\PYG{l+s+s1}{\PYGZsq{}}\PYG{p}{:}\PYG{l+s+s1}{\PYGZsq{}}\PYG{l+s+s1}{laminar}\PYG{l+s+s1}{\PYGZsq{}}\PYG{p}{,}
                    \PYG{l+s+s1}{\PYGZsq{}}\PYG{l+s+s1}{aspectAngle}\PYG{l+s+s1}{\PYGZsq{}}\PYG{p}{:}\PYG{l+m+mi}{90}\PYG{p}{,}
                    \PYG{l+s+s1}{\PYGZsq{}}\PYG{l+s+s1}{grooveDepth}\PYG{l+s+s1}{\PYGZsq{}}\PYG{p}{:}\PYG{l+m+mi}{15}\PYG{p}{,}
                    \PYG{l+s+s1}{\PYGZsq{}}\PYG{l+s+s1}{grooveRatio}\PYG{l+s+s1}{\PYGZsq{}}\PYG{p}{:}\PYG{l+m+mf}{0.65}\PYG{p}{,}\PYG{p}{\PYGZcb{}}
            \PYG{p}{)}
\end{sphinxVerbatim}

\sphinxAtStartPar
To change the gratings then, one can use the method implmented in the grating Ophyd device to change the grating, giving as
argument the pitch of the grating. To use the blazed grating use:

\begin{sphinxVerbatim}[commandchars=\\\{\}]
\PYG{n}{rp\PYGZus{}PG}\PYG{o}{.}\PYG{n}{change\PYGZus{}grating}\PYG{p}{(}\PYG{l+s+s1}{\PYGZsq{}}\PYG{l+s+s1}{1200}\PYG{l+s+s1}{\PYGZsq{}}\PYG{p}{)}
\end{sphinxVerbatim}

\sphinxAtStartPar
while to use the laminar grating:

\begin{sphinxVerbatim}[commandchars=\\\{\}]
\PYG{n}{rp\PYGZus{}PG}\PYG{o}{.}\PYG{n}{change\PYGZus{}grating}\PYG{p}{(}\PYG{l+s+s1}{\PYGZsq{}}\PYG{l+s+s1}{400}\PYG{l+s+s1}{\PYGZsq{}}\PYG{p}{)}
\end{sphinxVerbatim}

\sphinxAtStartPar
Four different kind of gratings can be implemented: \sphinxcode{\sphinxupquote{blazed}},
\sphinxcode{\sphinxupquote{laminar}}, \sphinxcode{\sphinxupquote{sinus}}, and \sphinxcode{\sphinxupquote{unknown}}. Each grating needs slightly different
parameters:

\begin{sphinxVerbatim}[commandchars=\\\{\}]
\PYG{n}{grating\PYGZus{}dict\PYGZus{}keys\PYGZus{}blazed} \PYG{o}{=} \PYG{p}{\PYGZob{}}\PYG{l+s+s1}{\PYGZsq{}}\PYG{l+s+s1}{name}\PYG{l+s+s1}{\PYGZsq{}}\PYG{p}{:}
                                \PYG{p}{\PYGZob{}}\PYG{l+s+s1}{\PYGZsq{}}\PYG{l+s+s1}{lineDensity}\PYG{l+s+s1}{\PYGZsq{}}\PYG{p}{:}\PYG{n}{value}\PYG{p}{,}
                                \PYG{l+s+s1}{\PYGZsq{}}\PYG{l+s+s1}{orderDiffraction}\PYG{l+s+s1}{\PYGZsq{}}\PYG{p}{:}\PYG{n}{value}\PYG{p}{,}
                                \PYG{l+s+s1}{\PYGZsq{}}\PYG{l+s+s1}{lineProfile}\PYG{l+s+s1}{\PYGZsq{}}\PYG{p}{:}\PYG{n}{value}\PYG{p}{,}
                                \PYG{l+s+s1}{\PYGZsq{}}\PYG{l+s+s1}{blazeAngle}\PYG{l+s+s1}{\PYGZsq{}}\PYG{p}{:}\PYG{n}{value}\PYG{p}{,}
                                \PYG{l+s+s1}{\PYGZsq{}}\PYG{l+s+s1}{aspectAngle}\PYG{l+s+s1}{\PYGZsq{}}\PYG{p}{:}\PYG{n}{value}\PYG{p}{,}
                                \PYG{p}{\PYGZcb{}}
                            \PYG{p}{\PYGZcb{}}
\PYG{n}{grating\PYGZus{}dict\PYGZus{}keys\PYGZus{}laminar} \PYG{o}{=} \PYG{p}{\PYGZob{}}\PYG{l+s+s1}{\PYGZsq{}}\PYG{l+s+s1}{name}\PYG{l+s+s1}{\PYGZsq{}}\PYG{p}{:}
                                \PYG{p}{\PYGZob{}}\PYG{l+s+s1}{\PYGZsq{}}\PYG{l+s+s1}{lineDensity}\PYG{l+s+s1}{\PYGZsq{}}\PYG{p}{:}\PYG{n}{value}\PYG{p}{,}
                                \PYG{l+s+s1}{\PYGZsq{}}\PYG{l+s+s1}{orderDiffraction:value}\PYG{l+s+s1}{\PYGZsq{}}\PYG{p}{,}
                                \PYG{l+s+s1}{\PYGZsq{}}\PYG{l+s+s1}{lineProfile}\PYG{l+s+s1}{\PYGZsq{}}\PYG{p}{:}\PYG{n}{value}\PYG{p}{,}
                                \PYG{l+s+s1}{\PYGZsq{}}\PYG{l+s+s1}{aspectAngle}\PYG{l+s+s1}{\PYGZsq{}}\PYG{p}{:}\PYG{n}{value}\PYG{p}{,}
                                \PYG{l+s+s1}{\PYGZsq{}}\PYG{l+s+s1}{grooveDepth}\PYG{l+s+s1}{\PYGZsq{}}\PYG{p}{:}\PYG{n}{value}\PYG{p}{,}
                                \PYG{l+s+s1}{\PYGZsq{}}\PYG{l+s+s1}{grooveRatio}\PYG{l+s+s1}{\PYGZsq{}}\PYG{p}{:}\PYG{n}{value}\PYG{p}{,}
                                \PYG{p}{\PYGZcb{}}
                            \PYG{p}{\PYGZcb{}}
\PYG{n}{grating\PYGZus{}dict\PYGZus{}keys\PYGZus{}sinus}   \PYG{o}{=} \PYG{p}{\PYGZob{}}\PYG{l+s+s1}{\PYGZsq{}}\PYG{l+s+s1}{name}\PYG{l+s+s1}{\PYGZsq{}}\PYG{p}{:}
                                \PYG{p}{\PYGZob{}}\PYG{l+s+s1}{\PYGZsq{}}\PYG{l+s+s1}{lineDensity}\PYG{l+s+s1}{\PYGZsq{}}\PYG{p}{:}\PYG{n}{value}\PYG{p}{,}
                                \PYG{l+s+s1}{\PYGZsq{}}\PYG{l+s+s1}{orderDiffraction}\PYG{l+s+s1}{\PYGZsq{}}\PYG{p}{:}\PYG{n}{value}\PYG{p}{,}
                                \PYG{l+s+s1}{\PYGZsq{}}\PYG{l+s+s1}{lineProfile}\PYG{l+s+s1}{\PYGZsq{}}\PYG{p}{:}\PYG{n}{value}\PYG{p}{,}
                                \PYG{l+s+s1}{\PYGZsq{}}\PYG{l+s+s1}{grooveDepth}\PYG{l+s+s1}{\PYGZsq{}}\PYG{p}{:}\PYG{n}{value}\PYG{p}{,}
                                \PYG{p}{\PYGZcb{}}
                            \PYG{p}{\PYGZcb{}}

\PYG{n}{grating\PYGZus{}dict\PYGZus{}keys\PYGZus{}unknown} \PYG{o}{=} \PYG{p}{\PYGZob{}}\PYG{l+s+s1}{\PYGZsq{}}\PYG{l+s+s1}{name}\PYG{l+s+s1}{\PYGZsq{}}\PYG{p}{:}
                                \PYG{p}{\PYGZob{}}\PYG{l+s+s1}{\PYGZsq{}}\PYG{l+s+s1}{lineDensity}\PYG{l+s+s1}{\PYGZsq{}}\PYG{p}{:}\PYG{n}{value}\PYG{p}{,}
                                \PYG{l+s+s1}{\PYGZsq{}}\PYG{l+s+s1}{orderDiffraction}\PYG{l+s+s1}{\PYGZsq{}}\PYG{p}{:}\PYG{n}{value}\PYG{p}{,}
                                \PYG{l+s+s1}{\PYGZsq{}}\PYG{l+s+s1}{lineProfile}\PYG{l+s+s1}{\PYGZsq{}}\PYG{p}{:}\PYG{n}{value}\PYG{p}{,}
                                \PYG{l+s+s1}{\PYGZsq{}}\PYG{l+s+s1}{gratingEfficiency}\PYG{l+s+s1}{\PYGZsq{}}\PYG{p}{:}\PYG{n}{value}
                                \PYG{p}{\PYGZcb{}}
                            \PYG{p}{\PYGZcb{}}
\end{sphinxVerbatim}

\sphinxstepscope


\chapter{API}
\label{\detokenize{API:api}}\label{\detokenize{API::doc}}

\section{Create Ophyd Devices from rml file}
\label{\detokenize{API:create-ophyd-devices-from-rml-file}}

\subsection{RaypyngOphydDevices}
\label{\detokenize{API:raypyngophyddevices}}\index{RaypyngOphydDevices (class in raypyng\_bluesky.RaypyngOphydDevices)@\spxentry{RaypyngOphydDevices}\spxextra{class in raypyng\_bluesky.RaypyngOphydDevices}}

\begin{fulllineitems}
\phantomsection\label{\detokenize{API:raypyng_bluesky.RaypyngOphydDevices.RaypyngOphydDevices}}
\pysigstartsignatures
\pysiglinewithargsret{\sphinxbfcode{\sphinxupquote{class\DUrole{w}{  }}}\sphinxcode{\sphinxupquote{raypyng\_bluesky.RaypyngOphydDevices.}}\sphinxbfcode{\sphinxupquote{RaypyngOphydDevices}}}{\emph{\DUrole{o}{*}\DUrole{n}{args}}, \emph{\DUrole{n}{RE}}, \emph{\DUrole{n}{rml\_path}}, \emph{\DUrole{n}{temporary\_folder}\DUrole{o}{=}\DUrole{default_value}{None}}, \emph{\DUrole{n}{name\_space}\DUrole{o}{=}\DUrole{default_value}{None}}, \emph{\DUrole{n}{prefix}\DUrole{o}{=}\DUrole{default_value}{None}}, \emph{\DUrole{n}{ray\_ui\_location}\DUrole{o}{=}\DUrole{default_value}{None}}, \emph{\DUrole{o}{**}\DUrole{n}{kwargs}}}{}
\pysigstopsignatures
\sphinxAtStartPar
Create ophyd devices from a RAY\sphinxhyphen{}UI rml file and adds them to a name space.

\sphinxAtStartPar
If you are using ipython \sphinxcode{\sphinxupquote{sys.\_getframe(0)}} returns the name space of the ipython instance.
(Remember to \sphinxcode{\sphinxupquote{import sys}})
\begin{quote}\begin{description}
\sphinxlineitem{Parameters}\begin{itemize}
\item {} 
\sphinxAtStartPar
\sphinxstyleliteralstrong{\sphinxupquote{RE}} (\sphinxstyleliteralemphasis{\sphinxupquote{RunEngine}}) \textendash{} Bluesky RunEngine

\item {} 
\sphinxAtStartPar
\sphinxstyleliteralstrong{\sphinxupquote{rml\_path}} (\sphinxstyleliteralemphasis{\sphinxupquote{str}}) \textendash{} the path to the rml file

\item {} 
\sphinxAtStartPar
\sphinxstyleliteralstrong{\sphinxupquote{temporary\_folder}} (\sphinxstyleliteralemphasis{\sphinxupquote{str}}) \textendash{} path where to create temporary folder. If None it is automatically
set into the ipython profile folder. Default to None.

\item {} 
\sphinxAtStartPar
\sphinxstyleliteralstrong{\sphinxupquote{name\_space}} (\sphinxstyleliteralemphasis{\sphinxupquote{frame}}\sphinxstyleliteralemphasis{\sphinxupquote{, }}\sphinxstyleliteralemphasis{\sphinxupquote{optional}}) \textendash{} If None the class will try to understand the correct namespace to add the Ophyd devices to.
If the automatic retrieval fails, pass \sphinxcode{\sphinxupquote{sys.\_getframe(0)}}. Defaults to None.

\item {} 
\sphinxAtStartPar
\sphinxstyleliteralstrong{\sphinxupquote{prefix}} (\sphinxstyleliteralemphasis{\sphinxupquote{str}}) \textendash{} the prefix to prepend to the oe names found in the rml file

\item {} 
\sphinxAtStartPar
\sphinxstyleliteralstrong{\sphinxupquote{ray\_ui\_location}} (\sphinxstyleliteralemphasis{\sphinxupquote{str}}) \textendash{} the location of the RAY\sphinxhyphen{}UI installation folder. If None the program will try to find it automatically. Deafault to None.

\end{itemize}

\end{description}\end{quote}
\index{append\_preprocessor() (raypyng\_bluesky.RaypyngOphydDevices.RaypyngOphydDevices method)@\spxentry{append\_preprocessor()}\spxextra{raypyng\_bluesky.RaypyngOphydDevices.RaypyngOphydDevices method}}

\begin{fulllineitems}
\phantomsection\label{\detokenize{API:raypyng_bluesky.RaypyngOphydDevices.RaypyngOphydDevices.append_preprocessor}}
\pysigstartsignatures
\pysiglinewithargsret{\sphinxbfcode{\sphinxupquote{append\_preprocessor}}}{}{}
\pysigstopsignatures
\sphinxAtStartPar
Add supplemental data to the RunEngine to trigger the simulations

\end{fulllineitems}

\index{create\_raypyng\_elements\_from\_rml() (raypyng\_bluesky.RaypyngOphydDevices.RaypyngOphydDevices method)@\spxentry{create\_raypyng\_elements\_from\_rml()}\spxextra{raypyng\_bluesky.RaypyngOphydDevices.RaypyngOphydDevices method}}

\begin{fulllineitems}
\phantomsection\label{\detokenize{API:raypyng_bluesky.RaypyngOphydDevices.RaypyngOphydDevices.create_raypyng_elements_from_rml}}
\pysigstartsignatures
\pysiglinewithargsret{\sphinxbfcode{\sphinxupquote{create\_raypyng\_elements\_from\_rml}}}{}{}
\pysigstopsignatures
\sphinxAtStartPar
Iterate through the raypyng objects created by RMLFile and create corresponding Ophyd Devices
\begin{quote}\begin{description}
\sphinxlineitem{Returns}
\sphinxAtStartPar
the Ophyd devices created

\sphinxlineitem{Return type}
\sphinxAtStartPar
OphydDevices

\end{description}\end{quote}

\end{fulllineitems}

\index{create\_trigger\_detector() (raypyng\_bluesky.RaypyngOphydDevices.RaypyngOphydDevices method)@\spxentry{create\_trigger\_detector()}\spxextra{raypyng\_bluesky.RaypyngOphydDevices.RaypyngOphydDevices method}}

\begin{fulllineitems}
\phantomsection\label{\detokenize{API:raypyng_bluesky.RaypyngOphydDevices.RaypyngOphydDevices.create_trigger_detector}}
\pysigstartsignatures
\pysiglinewithargsret{\sphinxbfcode{\sphinxupquote{create\_trigger\_detector}}}{}{}
\pysigstopsignatures
\sphinxAtStartPar
Create a trigger detector called RaypyngTriggerDetector

\end{fulllineitems}

\index{prepend\_to\_oe\_name() (raypyng\_bluesky.RaypyngOphydDevices.RaypyngOphydDevices method)@\spxentry{prepend\_to\_oe\_name()}\spxextra{raypyng\_bluesky.RaypyngOphydDevices.RaypyngOphydDevices method}}

\begin{fulllineitems}
\phantomsection\label{\detokenize{API:raypyng_bluesky.RaypyngOphydDevices.RaypyngOphydDevices.prepend_to_oe_name}}
\pysigstartsignatures
\pysiglinewithargsret{\sphinxbfcode{\sphinxupquote{prepend\_to\_oe\_name}}}{}{}
\pysigstopsignatures
\sphinxAtStartPar
Prepend a prefix to the name of all the Ophyd object created

\end{fulllineitems}

\index{trigger\_detector() (raypyng\_bluesky.RaypyngOphydDevices.RaypyngOphydDevices method)@\spxentry{trigger\_detector()}\spxextra{raypyng\_bluesky.RaypyngOphydDevices.RaypyngOphydDevices method}}

\begin{fulllineitems}
\phantomsection\label{\detokenize{API:raypyng_bluesky.RaypyngOphydDevices.RaypyngOphydDevices.trigger_detector}}
\pysigstartsignatures
\pysiglinewithargsret{\sphinxbfcode{\sphinxupquote{trigger\_detector}}}{}{}
\pysigstopsignatures
\sphinxAtStartPar
Return the RaypyngTriggerDetector
\begin{quote}\begin{description}
\sphinxlineitem{Returns}
\sphinxAtStartPar
the trigger detector

\sphinxlineitem{Return type}
\sphinxAtStartPar
{\hyperref[\detokenize{API:raypyng_bluesky.detector.RaypyngTriggerDetector}]{\sphinxcrossref{RaypyngTriggerDetector}}} ({\hyperref[\detokenize{API:raypyng_bluesky.detector.RaypyngTriggerDetector}]{\sphinxcrossref{RaypyngTriggerDetector}}})

\end{description}\end{quote}

\end{fulllineitems}


\end{fulllineitems}



\subsection{RaypyngDictionary}
\label{\detokenize{API:raypyngdictionary}}\index{RaypyngDictionary (class in raypyng\_bluesky.RaypyngOphydDevices)@\spxentry{RaypyngDictionary}\spxextra{class in raypyng\_bluesky.RaypyngOphydDevices}}

\begin{fulllineitems}
\phantomsection\label{\detokenize{API:raypyng_bluesky.RaypyngOphydDevices.RaypyngDictionary}}
\pysigstartsignatures
\pysiglinewithargsret{\sphinxbfcode{\sphinxupquote{class\DUrole{w}{  }}}\sphinxcode{\sphinxupquote{raypyng\_bluesky.RaypyngOphydDevices.}}\sphinxbfcode{\sphinxupquote{RaypyngDictionary}}}{\emph{\DUrole{o}{*}\DUrole{n}{args}}, \emph{\DUrole{o}{**}\DUrole{n}{kwargs}}}{}
\pysigstopsignatures
\sphinxAtStartPar
A class defining a dictionary of the differen elements in rayui and the classe to be used as Ophyd devices

\end{fulllineitems}



\section{Ophyd Signals}
\label{\detokenize{API:ophyd-signals}}

\subsection{RayPySignal}
\label{\detokenize{API:raypysignal}}\index{RayPySignal (class in raypyng\_bluesky.signal)@\spxentry{RayPySignal}\spxextra{class in raypyng\_bluesky.signal}}

\begin{fulllineitems}
\phantomsection\label{\detokenize{API:raypyng_bluesky.signal.RayPySignal}}
\pysigstartsignatures
\pysiglinewithargsret{\sphinxbfcode{\sphinxupquote{class\DUrole{w}{  }}}\sphinxcode{\sphinxupquote{raypyng\_bluesky.signal.}}\sphinxbfcode{\sphinxupquote{RayPySignal}}}{\emph{\DUrole{o}{*}\DUrole{n}{args}}, \emph{\DUrole{o}{**}\DUrole{n}{kwargs}}}{}
\pysigstopsignatures\index{get() (raypyng\_bluesky.signal.RayPySignal method)@\spxentry{get()}\spxextra{raypyng\_bluesky.signal.RayPySignal method}}

\begin{fulllineitems}
\phantomsection\label{\detokenize{API:raypyng_bluesky.signal.RayPySignal.get}}
\pysigstartsignatures
\pysiglinewithargsret{\sphinxbfcode{\sphinxupquote{get}}}{}{}
\pysigstopsignatures
\sphinxAtStartPar
The readback value

\end{fulllineitems}

\index{put() (raypyng\_bluesky.signal.RayPySignal method)@\spxentry{put()}\spxextra{raypyng\_bluesky.signal.RayPySignal method}}

\begin{fulllineitems}
\phantomsection\label{\detokenize{API:raypyng_bluesky.signal.RayPySignal.put}}
\pysigstartsignatures
\pysiglinewithargsret{\sphinxbfcode{\sphinxupquote{put}}}{\emph{\DUrole{o}{*}\DUrole{n}{args}}, \emph{\DUrole{o}{**}\DUrole{n}{kwargs}}}{}
\pysigstopsignatures
\sphinxAtStartPar
Put updates the internal readback value

\sphinxAtStartPar
The value is optionally checked first, depending on the value of force.
In addition, VALUE subscriptions are run.

\sphinxAtStartPar
Extra kwargs are ignored (for API compatibility with EpicsSignal kwargs
pass through).
\begin{quote}\begin{description}
\sphinxlineitem{Parameters}\begin{itemize}
\item {} 
\sphinxAtStartPar
\sphinxstyleliteralstrong{\sphinxupquote{value}} (\sphinxstyleliteralemphasis{\sphinxupquote{any}}) \textendash{} Value to set

\item {} 
\sphinxAtStartPar
\sphinxstyleliteralstrong{\sphinxupquote{timestamp}} (\sphinxstyleliteralemphasis{\sphinxupquote{float}}\sphinxstyleliteralemphasis{\sphinxupquote{, }}\sphinxstyleliteralemphasis{\sphinxupquote{optional}}) \textendash{} The timestamp associated with the value, defaults to time.time()

\item {} 
\sphinxAtStartPar
\sphinxstyleliteralstrong{\sphinxupquote{metadata}} (\sphinxstyleliteralemphasis{\sphinxupquote{dict}}\sphinxstyleliteralemphasis{\sphinxupquote{, }}\sphinxstyleliteralemphasis{\sphinxupquote{optional}}) \textendash{} Further associated metadata with the value (such as alarm status,
severity, etc.)

\item {} 
\sphinxAtStartPar
\sphinxstyleliteralstrong{\sphinxupquote{force}} (\sphinxstyleliteralemphasis{\sphinxupquote{bool}}\sphinxstyleliteralemphasis{\sphinxupquote{, }}\sphinxstyleliteralemphasis{\sphinxupquote{optional}}) \textendash{} Check the value prior to setting it, defaults to False

\end{itemize}

\end{description}\end{quote}

\end{fulllineitems}

\index{set() (raypyng\_bluesky.signal.RayPySignal method)@\spxentry{set()}\spxextra{raypyng\_bluesky.signal.RayPySignal method}}

\begin{fulllineitems}
\phantomsection\label{\detokenize{API:raypyng_bluesky.signal.RayPySignal.set}}
\pysigstartsignatures
\pysiglinewithargsret{\sphinxbfcode{\sphinxupquote{set}}}{\emph{\DUrole{n}{value}}}{}
\pysigstopsignatures
\sphinxAtStartPar
Set is like \sphinxtitleref{put}, but is here for bluesky compatibility
\begin{quote}\begin{description}
\sphinxlineitem{Returns}
\sphinxAtStartPar
\sphinxstylestrong{st} \textendash{} This status object will be finished upon return in the
case of basic soft Signals

\sphinxlineitem{Return type}
\sphinxAtStartPar
Status

\end{description}\end{quote}

\end{fulllineitems}


\end{fulllineitems}



\subsection{RayPySignalRO}
\label{\detokenize{API:raypysignalro}}\index{RayPySignalRO (class in raypyng\_bluesky.signal)@\spxentry{RayPySignalRO}\spxextra{class in raypyng\_bluesky.signal}}

\begin{fulllineitems}
\phantomsection\label{\detokenize{API:raypyng_bluesky.signal.RayPySignalRO}}
\pysigstartsignatures
\pysiglinewithargsret{\sphinxbfcode{\sphinxupquote{class\DUrole{w}{  }}}\sphinxcode{\sphinxupquote{raypyng\_bluesky.signal.}}\sphinxbfcode{\sphinxupquote{RayPySignalRO}}}{\emph{\DUrole{o}{*}\DUrole{n}{args}}, \emph{\DUrole{o}{**}\DUrole{n}{kwargs}}}{}
\pysigstopsignatures\index{put() (raypyng\_bluesky.signal.RayPySignalRO method)@\spxentry{put()}\spxextra{raypyng\_bluesky.signal.RayPySignalRO method}}

\begin{fulllineitems}
\phantomsection\label{\detokenize{API:raypyng_bluesky.signal.RayPySignalRO.put}}
\pysigstartsignatures
\pysiglinewithargsret{\sphinxbfcode{\sphinxupquote{put}}}{\emph{\DUrole{n}{value}}, \emph{\DUrole{o}{*}}, \emph{\DUrole{n}{timestamp}\DUrole{o}{=}\DUrole{default_value}{None}}, \emph{\DUrole{n}{force}\DUrole{o}{=}\DUrole{default_value}{False}}}{}
\pysigstopsignatures
\sphinxAtStartPar
Put updates the internal readback value

\sphinxAtStartPar
The value is optionally checked first, depending on the value of force.
In addition, VALUE subscriptions are run.

\sphinxAtStartPar
Extra kwargs are ignored (for API compatibility with EpicsSignal kwargs
pass through).
\begin{quote}\begin{description}
\sphinxlineitem{Parameters}\begin{itemize}
\item {} 
\sphinxAtStartPar
\sphinxstyleliteralstrong{\sphinxupquote{value}} (\sphinxstyleliteralemphasis{\sphinxupquote{any}}) \textendash{} Value to set

\item {} 
\sphinxAtStartPar
\sphinxstyleliteralstrong{\sphinxupquote{timestamp}} (\sphinxstyleliteralemphasis{\sphinxupquote{float}}\sphinxstyleliteralemphasis{\sphinxupquote{, }}\sphinxstyleliteralemphasis{\sphinxupquote{optional}}) \textendash{} The timestamp associated with the value, defaults to time.time()

\item {} 
\sphinxAtStartPar
\sphinxstyleliteralstrong{\sphinxupquote{metadata}} (\sphinxstyleliteralemphasis{\sphinxupquote{dict}}\sphinxstyleliteralemphasis{\sphinxupquote{, }}\sphinxstyleliteralemphasis{\sphinxupquote{optional}}) \textendash{} Further associated metadata with the value (such as alarm status,
severity, etc.)

\item {} 
\sphinxAtStartPar
\sphinxstyleliteralstrong{\sphinxupquote{force}} (\sphinxstyleliteralemphasis{\sphinxupquote{bool}}\sphinxstyleliteralemphasis{\sphinxupquote{, }}\sphinxstyleliteralemphasis{\sphinxupquote{optional}}) \textendash{} Check the value prior to setting it, defaults to False

\end{itemize}

\end{description}\end{quote}

\end{fulllineitems}

\index{set() (raypyng\_bluesky.signal.RayPySignalRO method)@\spxentry{set()}\spxextra{raypyng\_bluesky.signal.RayPySignalRO method}}

\begin{fulllineitems}
\phantomsection\label{\detokenize{API:raypyng_bluesky.signal.RayPySignalRO.set}}
\pysigstartsignatures
\pysiglinewithargsret{\sphinxbfcode{\sphinxupquote{set}}}{\emph{\DUrole{n}{value}}, \emph{\DUrole{o}{*}}, \emph{\DUrole{n}{timestamp}\DUrole{o}{=}\DUrole{default_value}{None}}, \emph{\DUrole{n}{force}\DUrole{o}{=}\DUrole{default_value}{False}}}{}
\pysigstopsignatures
\sphinxAtStartPar
Set is like \sphinxtitleref{put}, but is here for bluesky compatibility
\begin{quote}\begin{description}
\sphinxlineitem{Returns}
\sphinxAtStartPar
\sphinxstylestrong{st} \textendash{} This status object will be finished upon return in the
case of basic soft Signals

\sphinxlineitem{Return type}
\sphinxAtStartPar
Status

\end{description}\end{quote}

\end{fulllineitems}


\end{fulllineitems}



\section{Ophyd Axes}
\label{\detokenize{API:ophyd-axes}}

\subsection{Axes}
\label{\detokenize{API:axes}}\index{RaypyngAxis (class in raypyng\_bluesky.axes)@\spxentry{RaypyngAxis}\spxextra{class in raypyng\_bluesky.axes}}

\begin{fulllineitems}
\phantomsection\label{\detokenize{API:raypyng_bluesky.axes.RaypyngAxis}}
\pysigstartsignatures
\pysiglinewithargsret{\sphinxbfcode{\sphinxupquote{class\DUrole{w}{  }}}\sphinxcode{\sphinxupquote{raypyng\_bluesky.axes.}}\sphinxbfcode{\sphinxupquote{RaypyngAxis}}}{\emph{\DUrole{o}{*}\DUrole{n}{args}}, \emph{\DUrole{o}{**}\DUrole{n}{kwargs}}}{}
\pysigstopsignatures
\sphinxAtStartPar
The Axis used by all the Raypyng devices.

\sphinxAtStartPar
At the moment it is a comparator, in the future some other positioner will be used,
probably a SoftPositioner.
The class defines an empty dictionary, the \sphinxcode{\sphinxupquote{axes\_dict}} that will be then filled by each device.
\index{get() (raypyng\_bluesky.axes.RaypyngAxis method)@\spxentry{get()}\spxextra{raypyng\_bluesky.axes.RaypyngAxis method}}

\begin{fulllineitems}
\phantomsection\label{\detokenize{API:raypyng_bluesky.axes.RaypyngAxis.get}}
\pysigstartsignatures
\pysiglinewithargsret{\sphinxbfcode{\sphinxupquote{get}}}{}{}
\pysigstopsignatures
\sphinxAtStartPar
return the value of a certain axis as in the RMLFile
\begin{quote}\begin{description}
\sphinxlineitem{Returns}
\sphinxAtStartPar
the value of the axis in the RML file

\sphinxlineitem{Return type}
\sphinxAtStartPar
float

\end{description}\end{quote}

\end{fulllineitems}

\index{position (raypyng\_bluesky.axes.RaypyngAxis property)@\spxentry{position}\spxextra{raypyng\_bluesky.axes.RaypyngAxis property}}

\begin{fulllineitems}
\phantomsection\label{\detokenize{API:raypyng_bluesky.axes.RaypyngAxis.position}}
\pysigstartsignatures
\pysigline{\sphinxbfcode{\sphinxupquote{property\DUrole{w}{  }}}\sphinxbfcode{\sphinxupquote{position}}}
\pysigstopsignatures
\sphinxAtStartPar
The current position of the motor in its engineering units
:returns: \sphinxstylestrong{position}
:rtype: any

\end{fulllineitems}

\index{set() (raypyng\_bluesky.axes.RaypyngAxis method)@\spxentry{set()}\spxextra{raypyng\_bluesky.axes.RaypyngAxis method}}

\begin{fulllineitems}
\phantomsection\label{\detokenize{API:raypyng_bluesky.axes.RaypyngAxis.set}}
\pysigstartsignatures
\pysiglinewithargsret{\sphinxbfcode{\sphinxupquote{set}}}{\emph{\DUrole{n}{value}}}{}
\pysigstopsignatures
\sphinxAtStartPar
Write a value in the RMLFile for a certain element/axis
\begin{quote}\begin{description}
\sphinxlineitem{Parameters}
\sphinxAtStartPar
\sphinxstyleliteralstrong{\sphinxupquote{value}} (\sphinxstyleliteralemphasis{\sphinxupquote{float}}\sphinxstyleliteralemphasis{\sphinxupquote{,}}\sphinxstyleliteralemphasis{\sphinxupquote{int}}) \textendash{} the value to set to the axis

\end{description}\end{quote}

\end{fulllineitems}

\index{set\_axis() (raypyng\_bluesky.axes.RaypyngAxis method)@\spxentry{set\_axis()}\spxextra{raypyng\_bluesky.axes.RaypyngAxis method}}

\begin{fulllineitems}
\phantomsection\label{\detokenize{API:raypyng_bluesky.axes.RaypyngAxis.set_axis}}
\pysigstartsignatures
\pysiglinewithargsret{\sphinxbfcode{\sphinxupquote{set\_axis}}}{\emph{\DUrole{n}{obj}}, \emph{\DUrole{n}{axis}}}{}
\pysigstopsignatures
\sphinxAtStartPar
Set what axis should be used, based on the \sphinxcode{\sphinxupquote{axes\_dict}}
\begin{quote}\begin{description}
\sphinxlineitem{Parameters}\begin{itemize}
\item {} 
\sphinxAtStartPar
\sphinxstyleliteralstrong{\sphinxupquote{obj}} (\sphinxstyleliteralemphasis{\sphinxupquote{\_type\_}}) \textendash{} \_description\_

\item {} 
\sphinxAtStartPar
\sphinxstyleliteralstrong{\sphinxupquote{axis}} (\sphinxstyleliteralemphasis{\sphinxupquote{\_type\_}}) \textendash{} \_description\_

\end{itemize}

\end{description}\end{quote}

\end{fulllineitems}


\end{fulllineitems}



\subsection{SimulatedAxisSource}
\label{\detokenize{API:simulatedaxissource}}\index{SimulatedAxisSource (class in raypyng\_bluesky.axes)@\spxentry{SimulatedAxisSource}\spxextra{class in raypyng\_bluesky.axes}}

\begin{fulllineitems}
\phantomsection\label{\detokenize{API:raypyng_bluesky.axes.SimulatedAxisSource}}
\pysigstartsignatures
\pysiglinewithargsret{\sphinxbfcode{\sphinxupquote{class\DUrole{w}{  }}}\sphinxcode{\sphinxupquote{raypyng\_bluesky.axes.}}\sphinxbfcode{\sphinxupquote{SimulatedAxisSource}}}{\emph{\DUrole{o}{*}\DUrole{n}{args}}, \emph{\DUrole{o}{**}\DUrole{n}{kwargs}}}{}
\pysigstopsignatures
\sphinxAtStartPar
Define basic properties of the source.

\sphinxAtStartPar
Available axes:
\sphinxhyphen{} number of rays
\sphinxhyphen{} photon energy {[}eV{]}
\sphinxhyphen{} bandwidth {[}\% of photon energy{]}

\end{fulllineitems}



\subsection{class SimulatedAxisMisalign(RaypyngAxis):}
\label{\detokenize{API:class-simulatedaxismisalign-raypyngaxis}}\index{SimulatedAxisMisalign (class in raypyng\_bluesky.axes)@\spxentry{SimulatedAxisMisalign}\spxextra{class in raypyng\_bluesky.axes}}

\begin{fulllineitems}
\phantomsection\label{\detokenize{API:raypyng_bluesky.axes.SimulatedAxisMisalign}}
\pysigstartsignatures
\pysiglinewithargsret{\sphinxbfcode{\sphinxupquote{class\DUrole{w}{  }}}\sphinxcode{\sphinxupquote{raypyng\_bluesky.axes.}}\sphinxbfcode{\sphinxupquote{SimulatedAxisMisalign}}}{\emph{\DUrole{o}{*}\DUrole{n}{args}}, \emph{\DUrole{o}{**}\DUrole{n}{kwargs}}}{}
\pysigstopsignatures
\sphinxAtStartPar
Define misalignment axes
\begin{itemize}
\item {} 
\sphinxAtStartPar
translationXerror

\item {} 
\sphinxAtStartPar
translationYerror

\item {} 
\sphinxAtStartPar
translationZerror

\item {} 
\sphinxAtStartPar
rotationXerror

\item {} 
\sphinxAtStartPar
rotationYerror

\item {} 
\sphinxAtStartPar
rotationZerror

\end{itemize}

\end{fulllineitems}



\subsection{SimulatedAxisAperture}
\label{\detokenize{API:simulatedaxisaperture}}\index{SimulatedAxisAperture (class in raypyng\_bluesky.axes)@\spxentry{SimulatedAxisAperture}\spxextra{class in raypyng\_bluesky.axes}}

\begin{fulllineitems}
\phantomsection\label{\detokenize{API:raypyng_bluesky.axes.SimulatedAxisAperture}}
\pysigstartsignatures
\pysiglinewithargsret{\sphinxbfcode{\sphinxupquote{class\DUrole{w}{  }}}\sphinxcode{\sphinxupquote{raypyng\_bluesky.axes.}}\sphinxbfcode{\sphinxupquote{SimulatedAxisAperture}}}{\emph{\DUrole{o}{*}\DUrole{n}{args}}, \emph{\DUrole{o}{**}\DUrole{n}{kwargs}}}{}
\pysigstopsignatures
\sphinxAtStartPar
Define basic properties of the aperture,
the width and the height.

\end{fulllineitems}



\subsection{SimulatedAxisGrating}
\label{\detokenize{API:simulatedaxisgrating}}\index{SimulatedAxisGrating (class in raypyng\_bluesky.axes)@\spxentry{SimulatedAxisGrating}\spxextra{class in raypyng\_bluesky.axes}}

\begin{fulllineitems}
\phantomsection\label{\detokenize{API:raypyng_bluesky.axes.SimulatedAxisGrating}}
\pysigstartsignatures
\pysiglinewithargsret{\sphinxbfcode{\sphinxupquote{class\DUrole{w}{  }}}\sphinxcode{\sphinxupquote{raypyng\_bluesky.axes.}}\sphinxbfcode{\sphinxupquote{SimulatedAxisGrating}}}{\emph{\DUrole{o}{*}\DUrole{n}{args}}, \emph{\DUrole{o}{**}\DUrole{n}{kwargs}}}{}
\pysigstopsignatures
\sphinxAtStartPar
Define basic properties of the gratings:
\begin{itemize}
\item {} 
\sphinxAtStartPar
lineDensity

\item {} 
\sphinxAtStartPar
orderDiffraction

\item {} 
\sphinxAtStartPar
cFactor

\item {} 
\sphinxAtStartPar
lineProfile

\item {} 
\sphinxAtStartPar
blazeAngle

\item {} 
\sphinxAtStartPar
aspectAngle

\item {} 
\sphinxAtStartPar
grooveDepth

\item {} 
\sphinxAtStartPar
grooveRatio

\end{itemize}

\end{fulllineitems}



\section{Ophyd Detectors}
\label{\detokenize{API:ophyd-detectors}}

\subsection{Detector}
\label{\detokenize{API:detector}}\index{RaypyngDetector (class in raypyng\_bluesky.detector)@\spxentry{RaypyngDetector}\spxextra{class in raypyng\_bluesky.detector}}

\begin{fulllineitems}
\phantomsection\label{\detokenize{API:raypyng_bluesky.detector.RaypyngDetector}}
\pysigstartsignatures
\pysiglinewithargsret{\sphinxbfcode{\sphinxupquote{class\DUrole{w}{  }}}\sphinxcode{\sphinxupquote{raypyng\_bluesky.detector.}}\sphinxbfcode{\sphinxupquote{RaypyngDetector}}}{\emph{\DUrole{o}{*}\DUrole{n}{args}}, \emph{\DUrole{n}{information\_to\_extract}\DUrole{o}{=}\DUrole{default_value}{\textquotesingle{}intensity\textquotesingle{}}}, \emph{\DUrole{n}{parent\_detector\_name}\DUrole{o}{=}\DUrole{default_value}{None}}, \emph{\DUrole{o}{**}\DUrole{n}{kwargs}}}{}
\pysigstopsignatures
\sphinxAtStartPar
Defining a raypyng detector using a signal. The detector is created when
an Image Plane or an Image Plane Bundle is found in the rml file.
\index{get() (raypyng\_bluesky.detector.RaypyngDetector method)@\spxentry{get()}\spxextra{raypyng\_bluesky.detector.RaypyngDetector method}}

\begin{fulllineitems}
\phantomsection\label{\detokenize{API:raypyng_bluesky.detector.RaypyngDetector.get}}
\pysigstartsignatures
\pysiglinewithargsret{\sphinxbfcode{\sphinxupquote{get}}}{}{}
\pysigstopsignatures
\sphinxAtStartPar
The readback value

\end{fulllineitems}

\index{put() (raypyng\_bluesky.detector.RaypyngDetector method)@\spxentry{put()}\spxextra{raypyng\_bluesky.detector.RaypyngDetector method}}

\begin{fulllineitems}
\phantomsection\label{\detokenize{API:raypyng_bluesky.detector.RaypyngDetector.put}}
\pysigstartsignatures
\pysiglinewithargsret{\sphinxbfcode{\sphinxupquote{put}}}{\emph{\DUrole{n}{value}}, \emph{\DUrole{o}{*}}, \emph{\DUrole{n}{timestamp}\DUrole{o}{=}\DUrole{default_value}{None}}, \emph{\DUrole{n}{force}\DUrole{o}{=}\DUrole{default_value}{False}}}{}
\pysigstopsignatures
\sphinxAtStartPar
Put updates the internal readback value

\sphinxAtStartPar
The value is optionally checked first, depending on the value of force.
In addition, VALUE subscriptions are run.

\sphinxAtStartPar
Extra kwargs are ignored (for API compatibility with EpicsSignal kwargs
pass through).
\begin{quote}\begin{description}
\sphinxlineitem{Parameters}\begin{itemize}
\item {} 
\sphinxAtStartPar
\sphinxstyleliteralstrong{\sphinxupquote{value}} (\sphinxstyleliteralemphasis{\sphinxupquote{any}}) \textendash{} Value to set

\item {} 
\sphinxAtStartPar
\sphinxstyleliteralstrong{\sphinxupquote{timestamp}} (\sphinxstyleliteralemphasis{\sphinxupquote{float}}\sphinxstyleliteralemphasis{\sphinxupquote{, }}\sphinxstyleliteralemphasis{\sphinxupquote{optional}}) \textendash{} The timestamp associated with the value, defaults to time.time()

\item {} 
\sphinxAtStartPar
\sphinxstyleliteralstrong{\sphinxupquote{metadata}} (\sphinxstyleliteralemphasis{\sphinxupquote{dict}}\sphinxstyleliteralemphasis{\sphinxupquote{, }}\sphinxstyleliteralemphasis{\sphinxupquote{optional}}) \textendash{} Further associated metadata with the value (such as alarm status,
severity, etc.)

\item {} 
\sphinxAtStartPar
\sphinxstyleliteralstrong{\sphinxupquote{force}} (\sphinxstyleliteralemphasis{\sphinxupquote{bool}}\sphinxstyleliteralemphasis{\sphinxupquote{, }}\sphinxstyleliteralemphasis{\sphinxupquote{optional}}) \textendash{} Check the value prior to setting it, defaults to False

\end{itemize}

\end{description}\end{quote}

\end{fulllineitems}

\index{set() (raypyng\_bluesky.detector.RaypyngDetector method)@\spxentry{set()}\spxextra{raypyng\_bluesky.detector.RaypyngDetector method}}

\begin{fulllineitems}
\phantomsection\label{\detokenize{API:raypyng_bluesky.detector.RaypyngDetector.set}}
\pysigstartsignatures
\pysiglinewithargsret{\sphinxbfcode{\sphinxupquote{set}}}{\emph{\DUrole{n}{value}}, \emph{\DUrole{o}{*}}, \emph{\DUrole{n}{timestamp}\DUrole{o}{=}\DUrole{default_value}{None}}, \emph{\DUrole{n}{force}\DUrole{o}{=}\DUrole{default_value}{False}}}{}
\pysigstopsignatures
\sphinxAtStartPar
Set is like \sphinxtitleref{put}, but is here for bluesky compatibility
\begin{quote}\begin{description}
\sphinxlineitem{Returns}
\sphinxAtStartPar
\sphinxstylestrong{st} \textendash{} This status object will be finished upon return in the
case of basic soft Signals

\sphinxlineitem{Return type}
\sphinxAtStartPar
Status

\end{description}\end{quote}

\end{fulllineitems}

\index{trigger() (raypyng\_bluesky.detector.RaypyngDetector method)@\spxentry{trigger()}\spxextra{raypyng\_bluesky.detector.RaypyngDetector method}}

\begin{fulllineitems}
\phantomsection\label{\detokenize{API:raypyng_bluesky.detector.RaypyngDetector.trigger}}
\pysigstartsignatures
\pysiglinewithargsret{\sphinxbfcode{\sphinxupquote{trigger}}}{}{}
\pysigstopsignatures
\sphinxAtStartPar
Call that is used by bluesky prior to read()

\end{fulllineitems}


\end{fulllineitems}



\subsection{Trigger Detector}
\label{\detokenize{API:trigger-detector}}\index{RaypyngTriggerDetector (class in raypyng\_bluesky.detector)@\spxentry{RaypyngTriggerDetector}\spxextra{class in raypyng\_bluesky.detector}}

\begin{fulllineitems}
\phantomsection\label{\detokenize{API:raypyng_bluesky.detector.RaypyngTriggerDetector}}
\pysigstartsignatures
\pysiglinewithargsret{\sphinxbfcode{\sphinxupquote{class\DUrole{w}{  }}}\sphinxcode{\sphinxupquote{raypyng\_bluesky.detector.}}\sphinxbfcode{\sphinxupquote{RaypyngTriggerDetector}}}{\emph{\DUrole{o}{*}\DUrole{n}{args}}, \emph{\DUrole{n}{rml}}, \emph{\DUrole{n}{temporary\_folder}}, \emph{\DUrole{o}{**}\DUrole{n}{kwargs}}}{}
\pysigstopsignatures
\sphinxAtStartPar
The trigger detector is used to start the simulations. The simulations are done on
the machine where bluesky is running.
\index{get() (raypyng\_bluesky.detector.RaypyngTriggerDetector method)@\spxentry{get()}\spxextra{raypyng\_bluesky.detector.RaypyngTriggerDetector method}}

\begin{fulllineitems}
\phantomsection\label{\detokenize{API:raypyng_bluesky.detector.RaypyngTriggerDetector.get}}
\pysigstartsignatures
\pysiglinewithargsret{\sphinxbfcode{\sphinxupquote{get}}}{}{}
\pysigstopsignatures
\sphinxAtStartPar
The readback value

\end{fulllineitems}

\index{put() (raypyng\_bluesky.detector.RaypyngTriggerDetector method)@\spxentry{put()}\spxextra{raypyng\_bluesky.detector.RaypyngTriggerDetector method}}

\begin{fulllineitems}
\phantomsection\label{\detokenize{API:raypyng_bluesky.detector.RaypyngTriggerDetector.put}}
\pysigstartsignatures
\pysiglinewithargsret{\sphinxbfcode{\sphinxupquote{put}}}{\emph{\DUrole{n}{value}}, \emph{\DUrole{o}{*}}, \emph{\DUrole{n}{timestamp}\DUrole{o}{=}\DUrole{default_value}{None}}, \emph{\DUrole{n}{force}\DUrole{o}{=}\DUrole{default_value}{False}}}{}
\pysigstopsignatures
\sphinxAtStartPar
Put updates the internal readback value

\sphinxAtStartPar
The value is optionally checked first, depending on the value of force.
In addition, VALUE subscriptions are run.

\sphinxAtStartPar
Extra kwargs are ignored (for API compatibility with EpicsSignal kwargs
pass through).
\begin{quote}\begin{description}
\sphinxlineitem{Parameters}\begin{itemize}
\item {} 
\sphinxAtStartPar
\sphinxstyleliteralstrong{\sphinxupquote{value}} (\sphinxstyleliteralemphasis{\sphinxupquote{any}}) \textendash{} Value to set

\item {} 
\sphinxAtStartPar
\sphinxstyleliteralstrong{\sphinxupquote{timestamp}} (\sphinxstyleliteralemphasis{\sphinxupquote{float}}\sphinxstyleliteralemphasis{\sphinxupquote{, }}\sphinxstyleliteralemphasis{\sphinxupquote{optional}}) \textendash{} The timestamp associated with the value, defaults to time.time()

\item {} 
\sphinxAtStartPar
\sphinxstyleliteralstrong{\sphinxupquote{metadata}} (\sphinxstyleliteralemphasis{\sphinxupquote{dict}}\sphinxstyleliteralemphasis{\sphinxupquote{, }}\sphinxstyleliteralemphasis{\sphinxupquote{optional}}) \textendash{} Further associated metadata with the value (such as alarm status,
severity, etc.)

\item {} 
\sphinxAtStartPar
\sphinxstyleliteralstrong{\sphinxupquote{force}} (\sphinxstyleliteralemphasis{\sphinxupquote{bool}}\sphinxstyleliteralemphasis{\sphinxupquote{, }}\sphinxstyleliteralemphasis{\sphinxupquote{optional}}) \textendash{} Check the value prior to setting it, defaults to False

\end{itemize}

\end{description}\end{quote}

\end{fulllineitems}

\index{set() (raypyng\_bluesky.detector.RaypyngTriggerDetector method)@\spxentry{set()}\spxextra{raypyng\_bluesky.detector.RaypyngTriggerDetector method}}

\begin{fulllineitems}
\phantomsection\label{\detokenize{API:raypyng_bluesky.detector.RaypyngTriggerDetector.set}}
\pysigstartsignatures
\pysiglinewithargsret{\sphinxbfcode{\sphinxupquote{set}}}{\emph{\DUrole{n}{value}}, \emph{\DUrole{o}{*}}, \emph{\DUrole{n}{timestamp}\DUrole{o}{=}\DUrole{default_value}{None}}, \emph{\DUrole{n}{force}\DUrole{o}{=}\DUrole{default_value}{False}}}{}
\pysigstopsignatures
\sphinxAtStartPar
Set is like \sphinxtitleref{put}, but is here for bluesky compatibility
\begin{quote}\begin{description}
\sphinxlineitem{Returns}
\sphinxAtStartPar
\sphinxstylestrong{st} \textendash{} This status object will be finished upon return in the
case of basic soft Signals

\sphinxlineitem{Return type}
\sphinxAtStartPar
Status

\end{description}\end{quote}

\end{fulllineitems}

\index{trigger() (raypyng\_bluesky.detector.RaypyngTriggerDetector method)@\spxentry{trigger()}\spxextra{raypyng\_bluesky.detector.RaypyngTriggerDetector method}}

\begin{fulllineitems}
\phantomsection\label{\detokenize{API:raypyng_bluesky.detector.RaypyngTriggerDetector.trigger}}
\pysigstartsignatures
\pysiglinewithargsret{\sphinxbfcode{\sphinxupquote{trigger}}}{}{}
\pysigstopsignatures
\sphinxAtStartPar
Call that is used by bluesky prior to read()

\end{fulllineitems}


\end{fulllineitems}



\section{Ophyd Devices}
\label{\detokenize{API:ophyd-devices}}

\subsection{MisalignComponents}
\label{\detokenize{API:misaligncomponents}}\index{MisalignComponents (class in raypyng\_bluesky.devices)@\spxentry{MisalignComponents}\spxextra{class in raypyng\_bluesky.devices}}

\begin{fulllineitems}
\phantomsection\label{\detokenize{API:raypyng_bluesky.devices.MisalignComponents}}
\pysigstartsignatures
\pysiglinewithargsret{\sphinxbfcode{\sphinxupquote{class\DUrole{w}{  }}}\sphinxcode{\sphinxupquote{raypyng\_bluesky.devices.}}\sphinxbfcode{\sphinxupquote{MisalignComponents}}}{\emph{\DUrole{o}{*}\DUrole{n}{args}}, \emph{\DUrole{n}{obj}}, \emph{\DUrole{o}{**}\DUrole{n}{kwargs}}}{}
\pysigstopsignatures
\sphinxAtStartPar
Define the misalignment components of an optical element
using SimulatedAxisMisalign

\end{fulllineitems}



\subsection{SimulatedPGM}
\label{\detokenize{API:simulatedpgm}}\index{SimulatedPGM (class in raypyng\_bluesky.devices)@\spxentry{SimulatedPGM}\spxextra{class in raypyng\_bluesky.devices}}

\begin{fulllineitems}
\phantomsection\label{\detokenize{API:raypyng_bluesky.devices.SimulatedPGM}}
\pysigstartsignatures
\pysiglinewithargsret{\sphinxbfcode{\sphinxupquote{class\DUrole{w}{  }}}\sphinxcode{\sphinxupquote{raypyng\_bluesky.devices.}}\sphinxbfcode{\sphinxupquote{SimulatedPGM}}}{\emph{\DUrole{o}{*}\DUrole{n}{args}}, \emph{\DUrole{n}{obj}}, \emph{\DUrole{o}{**}\DUrole{n}{kwargs}}}{}
\pysigstopsignatures
\sphinxAtStartPar
Define the Plan Grating Monochromator using SimulatedAxisMisalign
and SimulatedAxisGrating.

\sphinxAtStartPar
Additionaly defines a two dictionaries to store different grating parameters
and a method  \sphinxcode{\sphinxupquote{change\_grating()}} to switch between them.
\index{change\_grating() (raypyng\_bluesky.devices.SimulatedPGM method)@\spxentry{change\_grating()}\spxextra{raypyng\_bluesky.devices.SimulatedPGM method}}

\begin{fulllineitems}
\phantomsection\label{\detokenize{API:raypyng_bluesky.devices.SimulatedPGM.change_grating}}
\pysigstartsignatures
\pysiglinewithargsret{\sphinxbfcode{\sphinxupquote{change\_grating}}}{\emph{\DUrole{n}{grating\_name}}}{}
\pysigstopsignatures
\sphinxAtStartPar
Change between gratings based on the line density
\begin{quote}\begin{description}
\sphinxlineitem{Parameters}
\sphinxAtStartPar
\sphinxstyleliteralstrong{\sphinxupquote{grating\_name}} (\sphinxstyleliteralemphasis{\sphinxupquote{str}}) \textendash{} the name you of the grating you want to use

\end{description}\end{quote}

\end{fulllineitems}

\index{rename\_default\_grating() (raypyng\_bluesky.devices.SimulatedPGM method)@\spxentry{rename\_default\_grating()}\spxextra{raypyng\_bluesky.devices.SimulatedPGM method}}

\begin{fulllineitems}
\phantomsection\label{\detokenize{API:raypyng_bluesky.devices.SimulatedPGM.rename_default_grating}}
\pysigstartsignatures
\pysiglinewithargsret{\sphinxbfcode{\sphinxupquote{rename\_default\_grating}}}{\emph{\DUrole{n}{new\_name}}}{}
\pysigstopsignatures
\sphinxAtStartPar
Rename the default grating
\begin{quote}\begin{description}
\sphinxlineitem{Parameters}
\sphinxAtStartPar
\sphinxstyleliteralstrong{\sphinxupquote{new\_name}} (\sphinxstyleliteralemphasis{\sphinxupquote{str}}) \textendash{} the new name for the default grating

\end{description}\end{quote}

\end{fulllineitems}

\index{rename\_grating() (raypyng\_bluesky.devices.SimulatedPGM method)@\spxentry{rename\_grating()}\spxextra{raypyng\_bluesky.devices.SimulatedPGM method}}

\begin{fulllineitems}
\phantomsection\label{\detokenize{API:raypyng_bluesky.devices.SimulatedPGM.rename_grating}}
\pysigstartsignatures
\pysiglinewithargsret{\sphinxbfcode{\sphinxupquote{rename\_grating}}}{\emph{\DUrole{n}{new\_name}}, \emph{\DUrole{n}{old\_name}}}{}
\pysigstopsignatures
\sphinxAtStartPar
Rename any grating
\begin{quote}\begin{description}
\sphinxlineitem{Parameters}\begin{itemize}
\item {} 
\sphinxAtStartPar
\sphinxstyleliteralstrong{\sphinxupquote{new\_name}} (\sphinxstyleliteralemphasis{\sphinxupquote{str}}) \textendash{} the new name for the default grating

\item {} 
\sphinxAtStartPar
\sphinxstyleliteralstrong{\sphinxupquote{old\_name}} (\sphinxstyleliteralemphasis{\sphinxupquote{str}}) \textendash{} the old name of the grating

\end{itemize}

\end{description}\end{quote}

\end{fulllineitems}


\end{fulllineitems}



\subsection{SimulatedApertures}
\label{\detokenize{API:simulatedapertures}}\index{SimulatedApertures (class in raypyng\_bluesky.devices)@\spxentry{SimulatedApertures}\spxextra{class in raypyng\_bluesky.devices}}

\begin{fulllineitems}
\phantomsection\label{\detokenize{API:raypyng_bluesky.devices.SimulatedApertures}}
\pysigstartsignatures
\pysiglinewithargsret{\sphinxbfcode{\sphinxupquote{class\DUrole{w}{  }}}\sphinxcode{\sphinxupquote{raypyng\_bluesky.devices.}}\sphinxbfcode{\sphinxupquote{SimulatedApertures}}}{\emph{\DUrole{o}{*}\DUrole{n}{args}}, \emph{\DUrole{n}{obj}}, \emph{\DUrole{o}{**}\DUrole{n}{kwargs}}}{}
\pysigstopsignatures
\sphinxAtStartPar
Define the apertures using SimulatedAxisMisalign
and SimulatedAxisAperture.

\end{fulllineitems}



\subsection{SimulatedMirror}
\label{\detokenize{API:simulatedmirror}}\index{SimulatedMirror (class in raypyng\_bluesky.devices)@\spxentry{SimulatedMirror}\spxextra{class in raypyng\_bluesky.devices}}

\begin{fulllineitems}
\phantomsection\label{\detokenize{API:raypyng_bluesky.devices.SimulatedMirror}}
\pysigstartsignatures
\pysiglinewithargsret{\sphinxbfcode{\sphinxupquote{class\DUrole{w}{  }}}\sphinxcode{\sphinxupquote{raypyng\_bluesky.devices.}}\sphinxbfcode{\sphinxupquote{SimulatedMirror}}}{\emph{\DUrole{o}{*}\DUrole{n}{args}}, \emph{\DUrole{n}{obj}}, \emph{\DUrole{o}{**}\DUrole{n}{kwargs}}}{}
\pysigstopsignatures
\sphinxAtStartPar
Define the mirrors using SimulatedAxisMisalign.

\end{fulllineitems}



\subsection{SimulatedSource}
\label{\detokenize{API:simulatedsource}}\index{SimulatedSource (class in raypyng\_bluesky.devices)@\spxentry{SimulatedSource}\spxextra{class in raypyng\_bluesky.devices}}

\begin{fulllineitems}
\phantomsection\label{\detokenize{API:raypyng_bluesky.devices.SimulatedSource}}
\pysigstartsignatures
\pysiglinewithargsret{\sphinxbfcode{\sphinxupquote{class\DUrole{w}{  }}}\sphinxcode{\sphinxupquote{raypyng\_bluesky.devices.}}\sphinxbfcode{\sphinxupquote{SimulatedSource}}}{\emph{\DUrole{o}{*}\DUrole{n}{args}}, \emph{\DUrole{n}{obj}}, \emph{\DUrole{o}{**}\DUrole{n}{kwargs}}}{}
\pysigstopsignatures
\sphinxAtStartPar
Define the source using SimulatedAxisSource.

\end{fulllineitems}



\section{Preprocessor}
\label{\detokenize{API:preprocessor}}

\subsection{MisalignComponents}
\label{\detokenize{API:id1}}\index{trigger\_sim() (in module raypyng\_bluesky.preprocessor)@\spxentry{trigger\_sim()}\spxextra{in module raypyng\_bluesky.preprocessor}}

\begin{fulllineitems}
\phantomsection\label{\detokenize{API:raypyng_bluesky.preprocessor.trigger_sim}}
\pysigstartsignatures
\pysiglinewithargsret{\sphinxcode{\sphinxupquote{raypyng\_bluesky.preprocessor.}}\sphinxbfcode{\sphinxupquote{trigger\_sim}}}{\emph{\DUrole{n}{plan}}, \emph{\DUrole{n}{trigger\_detector}}}{}
\pysigstopsignatures
\sphinxAtStartPar
Trigger simulations for raypyng plans

\sphinxAtStartPar
This function is composed of four steps:
\begin{description}
\sphinxlineitem{1\sphinxhyphen{} populate\_raypyng\_devices\_list\_at\_stage:}
\sphinxAtStartPar
at the ‘stage message’ each device is classified
and saved into two list. One list is dedicated to
raypng devices, and one for all the others

\sphinxlineitem{2\sphinxhyphen{} prepare\_simulations\_at\_open\_run:}
\sphinxAtStartPar
when the message is ‘open\_run’, if both raypyng
devices and normal devices have been staged raise
an exception. Otherwise the list of exports is
prepared(consists of detector names included in the plan)
and passed to the trigger detector.
The done simulation file is removed from the
temporary folder.

\sphinxlineitem{3\sphinxhyphen{}  insert\_before\_first\_det\_trigger:}
\sphinxAtStartPar
before the first detector is triggered, a trigger
message for the raypyng trigger detector is inserted
in the same group as the other detectors

\sphinxlineitem{4\sphinxhyphen{}  cleanup\_at\_close\_run:}
\sphinxAtStartPar
when the message is ‘close\_run’ the simulation\_done
file is removed and the list containing the raypyng and
other devices, created at point 1, are cleared.

\end{description}
\begin{quote}\begin{description}
\sphinxlineitem{Parameters}\begin{itemize}
\item {} 
\sphinxAtStartPar
\sphinxstyleliteralstrong{\sphinxupquote{plan}} (\sphinxstyleliteralemphasis{\sphinxupquote{bluesky.plan}}) \textendash{} the plan that is being executed

\item {} 
\sphinxAtStartPar
\sphinxstyleliteralstrong{\sphinxupquote{trigger\_detector}} ({\hyperref[\detokenize{API:raypyng_bluesky.detector.RaypyngTriggerDetector}]{\sphinxcrossref{\sphinxstyleliteralemphasis{\sphinxupquote{RaypyngTriggerDetector}}}}}) \textendash{} the trigger detector

\end{itemize}

\sphinxlineitem{Raises}\begin{itemize}
\item {} 
\sphinxAtStartPar
\sphinxstyleliteralstrong{\sphinxupquote{ValueError}} \textendash{} if in the plan a mix of raypyng devices are other devices

\item {} 
\sphinxAtStartPar
\sphinxstyleliteralstrong{\sphinxupquote{are used raise an exeption}} \textendash{} 

\end{itemize}

\end{description}\end{quote}

\end{fulllineitems}



\subsection{SupplementalDataRaypyng}
\label{\detokenize{API:supplementaldataraypyng}}\index{SupplementalDataRaypyng (class in raypyng\_bluesky.preprocessor)@\spxentry{SupplementalDataRaypyng}\spxextra{class in raypyng\_bluesky.preprocessor}}

\begin{fulllineitems}
\phantomsection\label{\detokenize{API:raypyng_bluesky.preprocessor.SupplementalDataRaypyng}}
\pysigstartsignatures
\pysiglinewithargsret{\sphinxbfcode{\sphinxupquote{class\DUrole{w}{  }}}\sphinxcode{\sphinxupquote{raypyng\_bluesky.preprocessor.}}\sphinxbfcode{\sphinxupquote{SupplementalDataRaypyng}}}{\emph{\DUrole{o}{*}\DUrole{n}{args}}, \emph{\DUrole{n}{trigger\_detector}}, \emph{\DUrole{o}{**}\DUrole{n}{kwargs}}}{}
\pysigstopsignatures
\sphinxAtStartPar
Supplemental data for raypyng.

\sphinxAtStartPar
The Run engine is needed to be able to include the
trigger detector automatically
\begin{quote}\begin{description}
\sphinxlineitem{Parameters}
\sphinxAtStartPar
\sphinxstyleliteralstrong{\sphinxupquote{trigger\_detector}} ({\hyperref[\detokenize{API:raypyng_bluesky.detector.RaypyngTriggerDetector}]{\sphinxcrossref{\sphinxstyleliteralemphasis{\sphinxupquote{RaypyngTriggerDetector}}}}}) \textendash{} The detector to trigger raypyng

\end{description}\end{quote}

\end{fulllineitems}




\renewcommand{\indexname}{Index}
\printindex
\end{document}